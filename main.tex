\documentclass[paper=a4, 12pt]{scrreprt}
%% Encoding UTF8
\usepackage[utf8]{inputenc}
%%Use Source Sans Pro Textstyle
\usepackage[default]{sourcesanspro}
%% 8 Bit Aufloesung der Buchstaben
\usepackage[T1]{fontenc}
%% Seitenraender
%\usepackage[scale=0.72]{geometry}
\usepackage[scale=0.72, twoside, bindingoffset=2mm]{geometry}

\usepackage[onehalfspacing]{setspace}

%% Spracheinstellungen
% \usepackage[english, naustrian]{babel} % your native language must be the last one!!
\usepackage[naustrian, english]{babel} % your native language must be the last one!!
%% erweiterte Farbenpalette
\usepackage[dvipsnames]{xcolor}
%% Abbildungen
\usepackage{graphicx}
%%Tabelen mit Farbe (cellcolor)
\usepackage{tabulary}
\usepackage{colortbl}
\PassOptionsToPackage{dvipsnames,svgnames,table}{xcolor}
%% Tabellen (erweitert)
\usepackage{tabularx}
%% TikZ + Circuit-TikZ (fuer Schaltungen)
\usepackage[europeanresistors, europeaninductors]{circuitikz}
%% Nuetzliche TikZ Libraries
\usetikzlibrary{arrows, automata, positioning}
%% mathematik
\usepackage{amsmath, amssymb}
%%Formelbeschreibung
\newenvironment{conditions}
  {\par\vspace{\abovedisplayskip}\noindent\begin{tabular}{>{$}l<{$} @{${}...{}$} l}}
  {\end{tabular}\par\vspace{\belowdisplayskip}}
%\usepackage{mathtools}	
%% pdf-einbindung
\usepackage{pdfpages}
%% scource-code einbindung
\usepackage{listings, scrhack} %scrhack vermeidet Umschaltung auf KOMA
% Floats..
\usepackage{courier}
%% euro-symbol
\usepackage{eurosym}
%% landcsape-seiten ermöglichen
\usepackage{lscape}

%% Todos
\setlength{\marginparwidth}{2cm}
\usepackage[]{todonotes}

%% Ganttdiagramme
\usepackage{pgfgantt}

%% Subfigures
\usepackage[lofdepth]{subfig}

%% Für Quellenangaben unter Bildern
\newcommand*{\quelle}[1]{\par\raggedleft\footnotesize Quelle:~#1}

%% Listings (Code)
\usepackage{listings}
\usepackage{color}

%% CUSTOM PACKAGES
\usepackage{tikz-timing}
\usepackage{pgfplots}
\usepackage{longtable}
\usepackage{mdframed}
\usepackage{booktabs}
\usepackage{multirow}
\usepackage{biblatex}
\usepackage{siunitx}
\usepackage{graphicx,wrapfig,lipsum}

\title{Letzte Hüfe}
\author{xB Fucking HELS}
\date{\today}

\begin{document}
    \maketitle
    \pagebreak

    \tableofcontents
    \pagebreak

    \chapter{Grundkonzepte}

\section{Grundeinheiten}
\begin{table}[!htb]
    \centering
    \begin{tabular}{|c|c|c|c|}
        \hline
        \textbf{Symbol} & \textbf{Bedeutung}        & \textbf{Einheit} & \textbf{Zusammenhang} \\ \hline
        U               & Spannung                  & Volt (V)         & -                     \\
        I               & Strom                     & Ampere (A)       & -                     \\
        R               & Widerstand                & Ohm ($\Omega$)   & -                     \\
        G               & Leitwert                  & Siemens (S)      & $\frac{1}{R}$         \\
        P               & Leistung                  & Watt (W)         & $U\cdot I$            \\
        C               & Kapazität                 & Farad (F)        & $C\cdot s$            \\
        Q               & Ladung                    & Coloumb (C)      & $C \cdot U$           \\
        L               & Induktivität              & Henry (H)        & -                     \\ \hline
        E               & Elektrische Feldstärke    & $\frac{V}{m}$    & $\frac{F}{Q}$         \\
        $\Psi$          & Elektrischer Fluss        & $C$              & -                     \\
        D               & Elektrische Flussdichte   & $\frac{C}{m^2}$  & $\frac{\Psi}{A^2}$    \\ \hline
    \end{tabular}
    \caption{Grundeinheiten}
    \label{Grundkonzepte/Einheiten/Tabelle}
\end{table}

\section{Konstanten}
\begin{table}[!htb]
    \centering
    \begin{tabular}{|c|c|c|}
        \hline
        \textbf{Symbol} & \textbf{Bedeutung}        & \textbf{Wert}                     \\ \hline
        $\epsilon_0$    & Permittivitätskonstante   & $8,854\cdot10^{-12}\frac{F}{m}$   \\
        $\mu_0$         & Permeabilitätskonstante   &                                   \\ \hline
    \end{tabular}
    \caption{Konstanten}
    \label{Grundkonzepte/Konstanten/Tabelle}
\end{table}

\pagebreak

    \chapter{Widerstand}

\section{Ohm'sches Gesetz}
Der Zusammenhang zwischen Spannung, Strom und Widerstand:
\begin{align}
    &U = R\cdot I       \\
    &I = \frac{U}{R}    \\
    &R = \frac{U}{I}
\end{align}

\section{Netzwerke}
\subsection{Serienschaltung}
\subsection{Parallelschaltung}
\section{Leitungswiderstand}
\section{Sterndreiecktransformation}
\section{Temperaturabhängigkeit}
\section{Potentiometer}

    \chapter{Kirchhoff}
Kirchhoff hat zwei fundamentale Regeln/Gesetze aufgestellt.

\section{Knotenregel}
Die Summe aller Ströme bei einem Knotenpunkt ist 0, d.h. Ströme die hineinfließen, müssen auch hinausfließen.

\section{Maschenregel}
Die Summer aller Spannungen in einer Masche ist 0.

$\sum U = 0$

\begin{align}
    &U_1 + U_2 = U_3        \\
    &U_1 + U_2 - U_3 = 0
\end{align}

Alle Spannungen in Richtung des Umlaufsinns: +
Alle Spannungen in Gegenrichtung des Umlaufsinns: -

    \chapter{Leistung}
\section{Blindleistung}
\subsection{Kompensation}
\section{Scheinleistung}
\section{Wirkleistung}
\section{Wirkungsgrad}

    \chapter{Quellen}

\section{Spannungsquelle}

\section{Stromquelle}

\section{Überlagerungsprinzip}

\section{Ersatzschaltbild}

    \chapter{Felder}

\section{Elektrisches Feld}

\section{Elektrischer Fluss}

\section{Magnetisches Feld}

\section{Magnetischer Fluss}

    \chapter{dB-Rechnung}
\section{dBV}
\section{dB}
\section{dBm}

    \chapter{Wechselstromtechnik}

\section{Komplexe Zahlen}

\section{Zeigerdiagramm}

\section{Impedanz}

\section{Admittanz}

    \chapter{Lineare Bauteile}

\section{Kondensator}

\subsection{Kapazität}

\subsection{Ladung}

\section{Spule}

\subsection{Induktionsvorgänge}

\subsection{Kopplungsgrad}

\subsection{Induktivitäten}

\section{RLC Netzwerke}

\subsection{$\tau$-Messung}

\section{Resonanzkreise}

\subsection{Güte}

\subsection{Bandbreite}

\section{Übertragungsfunktion}
Die Übertragungsfunktion beschreibt das Ausgangs- im Vergleich zum
Eingangssignal und ist definiert als
\begin{align}
    \underline{H}(j\omega)=\frac{\underline{U}_2}{\underline{U}_1}
\end{align}
wobei $\underline{U}_1$ der Eingang und $\underline{U}_2$ der Ausgang ist.

\subsection{Bodediagramm}
Das Bodediagramm zeigt das Verhalten eines Systems im logarithmischen
Frequenzbereich. Es besteht aus Amplitudengang (in dB) und Phasengang (in °)
und veranschaulicht die Übertragungsfunktion $H(j\omega)$.\\\\ Der
\textbf{Amplitudengang} zeigt, wie stark ein System das Ausgangssignal im
Vergleich zum Eingangssignal bei verschiedenen Frequenzen verstärkt oder
abschwächt. \\ Der \textbf{Phasengang} zeigt die Verzögerung oder Voreilung des
Ausgangssignals im Vergleich zum Eingangssignal bei verschiedenen Frequenzen.
\\\\ \textbf{Amplitudengang:}
\begin{center}
    \begin{tikzpicture}
        \begin{semilogxaxis}[
            xlabel={$\omega [rad]$},
            ylabel={$|\underline{H}(j\omega)|[dB]$},
            xmin=0.01, xmax=100,
            ymin=-40, ymax=10,
            xmode=log,
            grid=both,
            grid style={line width=.1pt, draw=gray!10},
            major grid style={line width=.2pt,draw=gray!50},
            minor tick num=5,
            width=10cm,
            height=6.5cm,
            ]

            \addplot[domain=0.01:100, blue, thick] {
                20 * log10(
                1/(x+1)
                )
            };
            \addlegendentry{Amplitude};
        \end{semilogxaxis}
    \end{tikzpicture}
\end{center}
\textbf{Phasengang:}
\begin{center}
    \begin{tikzpicture}
        \begin{semilogxaxis}[
                xlabel={$\omega [rad/s]$},
                ylabel={$\varphi [\degree]$},
                xmin=0.01, xmax=100,
                ymin=-100, ymax=0,
                xmode=log,
                grid=both,
                grid style={line width=.1pt, draw=gray!10},
                major grid style={line width=.2pt,draw=gray!50},
                minor tick num=5,
                width=10cm,
                height=6.5cm,
            ]

            \addplot[domain=0.01:100, red, thick] {-atan(x/1)};
            \addlegendentry{Phase};

        \end{semilogxaxis}
    \end{tikzpicture}
\end{center}

\section{Filter}
\subsection{Tiefpass}
Ein Tiefpassfilter lässt Signale mit niedrigen Frequenzen passieren, während es
Signale mit hohen Frequenzen blockiert oder abschwächt. Solche Filter können
durch LR- oder RC-Glieder aufgebaut werden.

\textbf{RC-Glied}
\begin{figure}
    \centering
    \includegraphics[width=0.5\textwidth]{LineareBauteile/RC-Glied.png}
    \caption{RC Tiefpass}
\end{figure}

\subsection{Hochpass}
\subsection{Allpass}


    \chapter{Halbleiter}

\section{Dioden}
\subsection{Sperrkennlinie}
\subsection{Durchbruchsspannung}

\section{MOSFET}

\section{Bipolartransistor}

    \chapter{OPV-Schaltungen}

\section{Verstärker}

\section{Schmitttrigger}

\section{Addierer \and Subtrahierer}

\section{Integrator \and Differenzierer}

\section{Instrumentation Amplifier}

    \chapter{Simulation}

\section{Altium}

\section{MicroCap}

\end{document}